\documentclass{article}
\usepackage{amsmath}
\usepackage{amssymb}
\usepackage{geometry}
\begin{document}

\title{\textsc{Courbes Param\'etr\'ees II}\footnote{Je recommande vivement le site \textsc{www.mathcurve.com} qui offre une grande collection de courbes.}}
\author{www.eleves.ens.fr/home/yhuang}
\date{17 Novembre, 2011}

\maketitle

\section{Tractrice}
\'Etudier la courbe d\'efinie par $\left\{\begin{array}{l}  x=t-th(t) \\ y=\frac{1}{ch(t)} \end{array}\right.$\\
On choisit un point $M$ quelconque et on trace la tangente au point $M$. Notons $P$ son intersection avec l'axe $(Ox)$. Montrer que $|PM|=1$. Quel est le mouvement du point $P$?

\section{Lemniscate de Bernouilli}
\'Etudier la courbe d'\'equation polaire $\rho=\sqrt{2\cos2\theta}$.\\
Quand est-ce que la droite d'\'equation $y=y_0$ coupe la courbe en deux points?\\
Montrer qu'avec $A=(-1,0)$ et $B=(1,0)$, pour tout $M$ appartenant \`a la courbe, $MA\times MB=1$.

\section{Cochl\'eo\"ide}
\'Etudier la courbe d'\'equation polaire $\rho=\frac{\sin\theta}{\theta}$.

\section{Quatratrice de Dinostrate}
\'Etudier la courbe d'\'equation polaire $\rho=\frac{\theta}{\sin\theta}$.

\section{Noeud de papillon}
\'Etudier la courbe d'\'equation polaire $\rho=\frac{\cos2\theta}{\cos^2\theta}$.

\section{Cisso\"ide droite}
\'Etudier la courbe d'\'equation polaire $\rho=\frac{\sin^2\theta}{cos\theta}$.\\
Une droite passant par l'origine coupe la courbe en un point $M$ puis l'aymptote en un point $P$. Soit $Q$ la projection de $P$ sur l'axe $(Oy)$. Montrer que les droites $QM$ et $OP$ sont orthogonales.

\section{Kampyle}
\'Etudier la courbe d'\'equation polaire $\rho=\frac{1}{\cos^2\theta}$.\\
Une droite passant par l'origine rencontre le cercle d'unit\'e en un point $M$ puis la courbe en un point $P$. Soit $Q$ la projection de $P$ sur l'axe $(Ox)$. Montrer que les droite $QM$ et $OP$ sont orthogonales.

\end{document}
