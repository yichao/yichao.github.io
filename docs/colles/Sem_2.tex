\documentclass{article}
\usepackage{amsmath}
\usepackage{amssymb}
\usepackage{geometry}
\begin{document}

\title{\textsc{Fonctions Usuelles}}
\author{www.eleves.ens.fr/home/yhuang}
\date{06 Octobre, 2011}

\maketitle

\section{Recollement}
R\'esoudre l'\'equation diff\'erentielle: $x(x+1)y'+y=\arctan x$.\\
R\'esoudre l'\'equation diff\'erentielle: $y'=|x-y|$.\\
R\'esoudre l'\'equation diff\'erentielle: $y''+|y|=1$ avec $y(0)=0$ et $y'(0)=1$.

\section{Changement de variables}
R\'esoudre l'\'equation diff\'erentielle: $y''-y'-e^{2x}y=e^{3x}$.\\
Indication: Posons $u=e^x$.\\
R\'esoudre l'\'equation diff\'erentielle: $x^2y''+4xy'+(2-x^2)y=1$.\\
Indication: Posons $y=\frac{y}{x^2}$.

\section{D\'eriver pour int\'egrer}
R\'esoudre l'\'equation $xy''-2y'-xy=0$.\\
R\'esoudre l'\'equation $f''(x)+f(-x)=x\cos x$ avec $f$ une fonction r\'eelle de classe $\mathcal{C}^2$.

\section{La fonction $\sin$ n'est pas rationnelle (X-ENS)}
Montrer que la fonction $\sin$ n'est pas rationnelle sur aucun intervalle r\'eel $[a,b]$.\\
On admettra le r\'esultat suivant: une fonction rationnelle n'a qu'un nombre fini de z\'ero sur un intervalle non vide $]a,b[$.\\
On rappelle que le degr\'e d'une fonction rationnelle $\frac{P}{Q}$ est d\'efini par $deg(P)-deg(Q)$ si $P\neq 0$ et $-\infty$ sinon. Cette d\'efinition ne d\'epend pas de la repr\'esentation choisie.

\end{document}
