\documentclass{article}
\usepackage{amsmath}
\usepackage{amssymb}
\usepackage{geometry}
\begin{document}

\setcounter{section}{9}

\title{\textsc{Fondements II}}
\author{www.eleves.ens.fr/home/yhuang}
\date{}
\maketitle

\subsection{Question du cours un peu longue}
Soit $E$ un ensemble (fini). On note $G$ l'ensemble des relations d'\'equivalence sur $E$. On dit qu'une relation d'\'equivalence $\sim$ est plus fine que $\cong$ si $\forall x,y\in E$, $x\sim y$, alors $x\cong y$.\\
Montrer que la relation ``est plus fine que'' est une relation d'ordre sur $G$.\\
\'El\'ement maximal? \'El\'ement minimal?

\subsection{Quotienter pour s'en sortir}
On se donne une relation $\mathcal{R}$ r\'eflexive et transitive, mais non n\'ecessairement antisym\'etrique sur un ensemble $E$. Le but de cet exercice est de construire une relation d'\'equivalence telle qu'en quotientant par cette \'equivalence, la relaion $\mathcal{R}$ devienne une relation d'ordre.\\
\textsf{Exemple: Consid\'erons l'ensemble $\mathbb{Z}$ muni d'une relation d\'efinie par $a\mathcal{R}b$ ssi $|a|\leq|b|$, avec $a,b$ entiers.\\
Montrer que cette relation est r\'eflexive et transitive, mais non n\'ecessairement antisym\'etrique.\\
On d\'efinie ensuite la relation $\sim$ par $a\sim b$ ssi $|a|=|b|$, avec $a,b$ entiers.\\
Montrer que cette relation est une relation d'\'equivalence. Commenter ses classes d'\'equivalence.\\
Montrer que la relation $\mathcal{R}$ devient une relation d'ordre sur $\mathbb{Z}/\sim$.\\}
Dans le cas g\'en\'eral, on peut d\'efinir une relation d'\'equivalence (v\'erifier-le) $x\sim y$ ssi $x\mathcal{R}y$ et $y\mathcal{R}x$.\\
Montrer alors que $\mathcal{R}$ est une relation d'ordre sur $E/\sim$. On devrait d'abord d\'ecrire la forme de $\mathcal{R}$ sur $E/\sim$ (la relation induite), et montrer qu'elle est ind\'ependente des repr\'esentants choisis.

\subsection{Partitions et relations d'\'equivalence}
Montrer qu'une relation d'\'equivalence d\'efinit une partition sur un ensemble par ses classes d'\'equivalence.\\
Montrer qu'en se donnant une partition, on peut trouver une relation d'\'equivalence telle que ses classes d'\'equivalence fassent exactement la partition donn\'ee.

\subsection{Cauchy-Schwarz, le retour}
1) Dans cette question, on va repr\'esenter une relation binaire par un tableau. Soient donc $E$ un ensemble muni d'un relation binaire $\mathcal{R}$, et le tableau $E\times E$ que je vais dessiner au tableau. Maintenant pour chaque case, on peut soit mettre un $0$, soit mettre un $1$.\\
Dire pourquoi ce tableau d\'etermine compl\`etement la relation binaire et r\'eciproquement.\\
Quelle serait la t\^ete du tableau d'une relation r\'eflexive? D'une relation sym\'etrique? D'une relation sym\'etrique et transitive? D'une relation d'\'equivalence?\\
2) Maintenant, supposons que $\mathcal{R}$ soit une relation d'\'equivalence. On note $|\mathcal{R}|$ la somme de toutes les cases du tableau de $\mathcal{R}$, i.e. le nombre de $1$ dans le tableau. On note aussi $|E/\mathcal{R}|$ le nombre de classes d'\'equivalence de $\mathcal{R}$\footnote{La d\'efinition du cardinal n'a pas encore \'et\'e vue\dots}. Montrer alors que $|E|^2\leq|E/\mathcal{R}|\times|\mathcal{R}|$.

\subsection{Propri\'et\'e universelle du quotient}
Soient $E,F$ deux ensembles et $\sim$ une relation d'\'equivalence sur $E$. On note, pour $a\in E$, $[a]$ sa classe d'\'equivalence, et $\pi:a\mapsto[a]$ l'application qui a un \'el\'ement associe sa classe d'\'equivalence.\\
On suppose de plus qu'on a une application $f:E\to F$ qui est constante sur chaque classe d'\'equivalence, i.e. si $a,b\in E$, $a\sim b$ alors $f(a)=f(b)$.\\
Montrer qu'il existe une unique application $g$ telle que $f=g\circ\pi$.

\subsection{Pi\`eges\dots}
1) L'application qui \`a deux ensembles $E$ et $F$ associe $E\cup F$ est-elle une loi de composition interne?\\
2) Dans la d\'efinition d'une relation d'\'equivalence, il y a trois conditions.\\
Alice dit que la r\'eflexivit\'e est une cons\'equence de la sym\'etrie et la transitivit\'e. Elle dit que si $a\sim b$, alors $b\sim a$ par sym\'etrie et $a\sim b\sim a$ donc $a\sim a$ par transitivit\'e.\\
A-t-elle raison? Peut-on retirer la r\'eflexivit\'e dans la d\'efinition?

\subsection{$(E^{E})^{E}$ et $E^{E\times E}$}
Montrer que se donner une loi de composition interne sur un ensemble $E$, c'est la m\^eme chose que de se donner une application de $E$ dans $E^{E}$.

\subsection{\'Equivalence sur de relations d'\'equivalence}
Soient $E$ un ensemble fini et $G$ l'ensemble des relations d'\'equivalences sur $E$. Pour une relation d'\'equivalence $u$, on note $[u]$ l'ensemble de ses classes d'\'equivalence (non vides).\\
On dit dans cette colle que la relation $v$ est un ``(1-)raffinement'' de $u$ s'il existe $e\in E$ et $X\in [u]$ tels que $e\in X$, $\{e\}\neq X$ et $[v]=\{C,\{e\},X-\{e\}|C\in[u], C\neq X\}$.\\
Ceci engendre une relation d'\'equivalence $\sim$ sur $G$. Plus pr\'ecis\'ement, on dit que $u\sim v$ ssi il existe une suite $\{u=u_0,\dots,u_n=v\}$ telle que $\forall 1\leq i\leq n$, $u_{i-1}$ est un (1-)raffinement de $u_i$ ou $u_i$ est un (1-)raffinement de $u_{i-1}$.\\
Montrer que $\sim$ est une relation d'\'equivalence triviale.

\subsection{$\mathbb{C}$}
On consid\`ere $\mathbb{R}[X]$, l'ensemble des polyn\^omes \`a une variable $X$ \`a coefficient dans $\mathbb{R}$.\\
On d\'efinit la relation d'\'equivalence suivante: $\forall P,Q\in\mathbb{R}[X]$, $P\sim Q$ ssi $X^2+1|P-Q$. On note l'ensemble quotient $\mathbb{R}[X]/(X^2+1)$.\\
0) V\'erifier que c'est une relation d'\'equivalence.\\
1) Montrer que tout polyn\^ome est \'equivalent \`a un polyn\^ome de degr\'e au plus $1$.\\
2) Calculer $(a+bX)+(c+dX)$, $(a+bX)(c+dX)$ dans le quotient.\\
3) Que dire de l'application $[X]\mapsto i$, $\mathbb{R}[X]/(X^2+1)\to\mathbb{C}$?

\end{document} 