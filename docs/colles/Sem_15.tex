\documentclass{article}
\usepackage{amsmath}
\usepackage{amssymb}
\usepackage{geometry}
\begin{document}

\setcounter{section}{15}

\title{\textsc{Limites, continuit\'e}}
\author{www.eleves.ens.fr/home/yhuang}
\date{}
\maketitle

\subsection{IMC}
\subsubsection{IMC 2009, 1}
Soient $f$ et $g$ deux fonctions r\'eelles telles que $f(r)\leq g(r)$ pour tout nombre rationnel $r$. Est-ce que $f(x)\leq g(x)$ pour tout $x\in\mathbb{R}$ si on suppose de plus que:\\
1) $f$ et $g$ sont croissante.\\
2) $f$ et $g$ sont continues.

\subsubsection{IMC 2011, 1}
Soit $f:\mathbb{R}\to\mathbb{R}$ une fonction continue. On dit qu'un point $x$ est ``sombre'' s'il existe une point $y\in\mathbb{R}, y>x$ tel que $f(y)>f(x)$. Soient $a<b$ deux nombres r\'eels, et supposons que:\\
1) Tous les points dans l'intervalle $]a,b[$ sont sombres;\\
2) $a$ et $b$ ne sont pas sombres.\\
Montrer que $f(a)=f(b)$.

\subsubsection{IMC 2006, 2}
\emph{Pr\'erequis}: une fonction $1$-lipschizienne est continue.\\
Trouver toutes les fonctions $f:\mathbb{R}\to\mathbb{R}$ telles que pour tout couple r\'eel $(a,b)$, l'image de $f([a,b])$ soit un intervalle ferm\'e de longueur $b-a$.

\subsection{Une famille de fonctions continues}
Soit $f:\mathbb{R}^+\to\mathbb{R}$ une fonction croissante telle que $x\mapsto\frac{f(x)}{x}$ soit d\'ecroissante. Montrer que $f$ est continue.

\subsubsection{Si $f^n$ admet un point fixe\dots}
Soit $f$ une fonction r\'eelle continue telle que $f^n$ admette un point fixe. Montrer que $f$ admet un point fixe.

\subsection{Fonction dilatante}
On dit qu'une fonction $f:\mathbb{R}\to\mathbb{R}$ est dilatante si $\forall (x,y)\in\mathbb{R}^2$, $|f(x)-f(y)|\geq|x-y|$.\\
\textbf{1.} Donner un exmple de fonction dilatante non monotone.\\
Dans la suite, on suppose de plus que $f$ est continue.\\
\textbf{2.} Montrer que $f$ est strictement monotone, puis $f$ est bijective.\\
\textbf{3.} On suppose qu'il existe un intervalle r\'eel $[a,b]$ stable par $f$. Montrer que $f$ admet un point fixe.\\
Si $f$ est croissante, montrer que $f$ est identit\'e sur $[a,b]$.\\
\textbf{4.} Soit $A$ l'ensemble des points fixes de $f$. Montrer que si $A$ est un intervalle ferm\'e ou vide.\\
Supposons $A$ non vide. On se donne ensuite une suite r\'eelle $(x_n)_{n\in\mathbb{N}}$ telle que $\forall n\in\mathbb{N}, x_{n+1}=f^{-1}(x_n)$. Montrer que $(x_n)$ est constante ou elle converge vers une extr\'emit\'e de $A$.

\subsection{Fonction continue de [0,1] dans [0,1]}
Montrer qu'une fonction continue $f$ de $[0,1]$ dans $[0,1]$ admet un point fixe.\\
Les deux questions suivantes sont ind\'ependantes.\\
Montrer que si $f,g$ sont deux telles fonctions telles que $f\circ g=g\circ f$, alors il existe un point $x\in[0,1]$ tel que $f(x)=g(x)$.\\
Montrer qu'il exite une suite $(u_n)_{n\in\mathbb{N}}$ telle que $f(u_n)=u_n^n$. Si on suppose de plus que $f$ est strictement d\'ecroissante, montrer que $\forall n\geq 1$, $u_n$ est unique et limite de $\lim\limits_{n\to\infty}u_n=1$.

\subsection{Conjugaison des hom\'eomorphismes de [0,1] dans [0,1] (X-ENS)}
Soit $G$ l'ensemble des hom\'omorphismes de $[0,1]$ sur $[0,1]$. Soit $f$ (resp. $g$) un \'el\'ement de $G$ tels que les seuls points fixes pour $f$ (resp, $g$) sont $\{0,1\}$. Montrer que $f$ et $g$ sont conjugu\'es dans $G$, i.e. il existe un \'el\'ement $h$ de $G$ tel que $f\circ h=h\circ g$.

\subsection{Module de continuit\'e\protect\footnote{Un peu de connaissance sur l'uniforme continuit\'e aidera \`a mieux comprendre la situation.}}
On note $\omega_{f}(\delta)=\sup\{|f(x)-f(y)|, |x-y|\leq\delta\}$ si $f$ est une fonction continue sur l'intervalle $[a,b]$.\\
Montrer que $\omega_{f}$ est une fonction continue.

\subsection{Nombres de rotation du tore (X-ENS)}
0) Montrer qu'une application continue de $\mathbb{R}$ vers $\mathbb{Z}$ est constante.\\
Soit $f:\mathbb{R}\to\mathbb{R}$ une fonction continue croissante, telle que $\forall x\in\mathbb{R}, f(x+1)=f(x)+1$. On va montrer que $\frac{f^n(x)}{n}$ admet une limite ind\'ependante de $x$.\\
$\alpha$) Soit $(x,y)\in\mathbb{R}^2$. Montrer qu'il existe $k\in\mathbb{N}$ tel que $\forall n\in\mathbb{N}$, $|f^n(x)-f^n(y)|\leq k$.\\
$\beta$)  Montrer que pour tout $(n,m)\in\mathbb{N}^2$, $f^n(0)+f^m(0)-1\leq f^{n+m}(0)\leq f^n(0)+f^m(0)+1$.\\
$\gamma$) V\'erifier que la limite de la suite $(u_n(0))_{n\in\mathbb{N}}$ existe et appartient \`a $[f(0)-1,f(0)+1]$. Conclure. 

\subsection{Courbe de Peano}

\end{document}