\documentclass{article}
\usepackage{amsmath}
\usepackage{amssymb}
\usepackage{geometry}
\begin{document}

\setcounter{section}{12}

\title{\textsc{Groupes, Anneaux, Corps: I}\protect\footnote{Les anneaux sont suppos\'es commutatifs unitaires.}}
\author{www.eleves.ens.fr/home/yhuang}
\date{}
\maketitle

\subsection{Exemples g\'en\'eraux}
\textbf{$\bullet$ Une bijection.}
Soit $G$ un groupe et $g\in G$. Montrer que l'application $a\mapsto ga$ est une bijection, que l'on appelle la translation (\`a gauche) par $g$.\\
\textbf{$\bullet$ Un automorphisme.}
Soit $G$ un groupe et $g\in G$. Montrer que l'application $a\mapsto gag^{-1}$ est un endomorphisme, que l'on appelle plus souvent l'automorphisme int\'erieur (associ\'e \`a $g$).\\
Montrer que si $H$ est un sous-groupe de $G$, alors $H'=gHg^{-1}$ est aussi un sous-groupe de $G$. On dit dans ce cas que $H$ et $H'$ sont conjugu\'es.\\
\textbf{$\bullet$ Un sous-groupe.}
Soit $G$ un groupe et $g\in G$. Montrer que l'ensemble des \'el\'ements de $G$ qui commutent avec $g$ est un sous-groupe de $G$. Voir \ref{normalisateur}.\\
\textbf{$\bullet$ Un autre sous-groupe.}
Soit $G$ un groupe et $g\in G$. Montrer que l'ensemble $\{g^n|n\in\mathbb{Z}\}$ est un sous-groupe de $G$.\\
\textbf{$\bullet$ Groupes d'ordres petits.}
D\'ecrire tous les groupes d'ordre inf\'erieur \`a $6$.


\subsection{Exemples de groupes}
\subsubsection{Groupes cycliques}
1) Montrer que tout groupe cyclique est un groupe ab\'elien.\\
2) Lang, P24\dots
\subsubsection{Groupes di\'edraux}
\subsubsection{Groupes libres (Cayley?)}

\subsection{Exemples de sous-groupes}
\subsubsection{Centre}
*) $G/Z(G)\cong Int(G)$.
\subsubsection{Centralisateurs}
Propri\'et\'es\dots
\subsubsection{Normalisateurs\label{normalisateur}}

\subsection{Exercices sur les groupes}
\subsubsection{Construction d'un groupe \`a partir d'un mono\"ide}
\subsubsection{Th\'eor\`eme de Lagrange}
\subsubsection{Th\'eor\`eme de factorisation}
Soit $f:G\to G'$ un morphisme de groupes.\\
1) Montrer que $Ker(f)$ est un sous-groupe distingu\'e de $G$.\\
Soient $H$ un sous-groupe distingu\'e de $G$ et un morphisme de groupes $f:G\to G'$ tel que $f(H)=e_{G'}$ (autrement dit $H\subset Ker(f)$).\\
2) Montrer qu'il existe un morphisme de groupe $g:G/H\to G'$ tel que $f=g\circ p$, o\`u $p: G\to G/H$ est la projection canonique.

\subsection{Exemples de morphismes de groupes}

\subsection{Exercices sur les morphismes de groupes}
\subsubsection{Th\'eor\`eme de factorisation}
1) Montrer que si $f:G\to G'$ est un morphisme de groupes, alors $Ker(f)$ est un sous-groupe de $G$.\\
1,5) Montrer que si $G'$ est ab\'elien, alors le noyau de $f$ est un sous-groupe distingu\'e de $G$.\\
2) Montrer que si $H$ est un sous-groupe de $G$, alors il existe un morphisme de groupes $f:G\to G/H$ canonique tel que $H$ soit le noyau de $f$.
\subsubsection{Diagrammes commutatifs et suites exactes}
D\'efinitions.\\
0) Montrer que si on a une suite exacte courte $0\to E\to F\to 0$, alors $E$ et $F$ sont isomorphes.\\
1) Consid\'erons un diagramme commutatif de groupes ab\'eliens\dots\footnote{Bourbaki, Alg\`ebre commutatives, P17.}
\subsubsection{Lemme des cinq}
Dessin.\\
Montrer que si la premi\`ere, la seconde, la quatri\`eme et la cinqui\`eme fl\`eches sont des isomorphismes, alors la troi\`eme fl\`eche l'est aussi.

\end{document}