\documentclass{article}
\usepackage{amsmath}
\usepackage{amssymb}
\begin{document}
\renewcommand\partname{Partie}

\begin{center}
\Huge{\textbf{Nombres Complexes}}
\end{center}
\begin{center}
|\\
\end{center}
\section{Formule de De Moivre}
D\'emontrer la formule de De Moivre: $(\cos\phi+i\sin\phi)^n=\cos n\phi+i\sin n\phi$.\\
Rappeler les interpr\'etations graphiques pour les additions et multiplications complexes.\\
Faire l'interpr\'etation graphique de la formule de De Moivre.
\section{Racines de l'unit\'e}
Soit $n$ un entier strictement positif. Situer les racines $n$-i\`eme de l'unit\'e sur la boule unit\'e du plan complexe.\\
\`A l'aide de la figure, conjecturer la somme de toutes les racines de l'unit\'e.\\
D\'emontrer proprement la conjecture.\\
Que peut-on dire de leur produit?
\section{Racines complexes d'un polyn\^ome r\'eel de degr\'e impair}
Soit $P$ un polyn\^ome \`a coefficients r\'eels.\\
Soit $x$ une racine complexe de $P$, i.e. $P(x)=0$.\\
Montrer que le conjugu\'e de $x$, $\overline{x}$, est aussi une racine complexe de $P$.\\
Que peut-on dire sur le nombre de racines complexes non r\'eels de ce polyn\^ome?\\
On admet le th\'eor\`eme fondamental de l'alg\`ebre, alors un polyn\^ome de degr\'e $n\neq 0$ admet $n$ racines complexes \'eventuellement confondues.\\
Montrer que $P$ admet une racine r\'eelle.
\section{Exponentielle complexe}
Montrer que l'exponentielle complexe n'est pas injective.
\begin{center}
|\\
\end{center}
\section{In\'egalit\'e de Bell}
Montrer qu'il existe des points $a$, $a'$, $b$, $b'$ sur la sph\'ere unit\'e de l'espace tels que $(<a,b>+<b,a'>+<a',b'>-<b',a>)\geq 2$ (on confond ici les points avec leurs coordon\'ees).\\
Int\'er\^et: Pour montrer que Monsieur Einstein a eu tort quand il a postul\'e l'existence de variables cach\'ees.
\section{Demi-plan complexe}
On d\'efinit $\mathbb{H}=\{z\in\mathbb{C}\mid Im~z>0\}$ le demi-plan du plan complexe.\\
Montrer que $z\in\mathbb{H}$ ssi $-\frac{1}{z}\in\mathbb{H}$.\\
On se donne $z,a$ deux nombres complexes.\\
Montrer que $|1-z\overline{a}|^2-|z-a|^2=(1-|z|^2)(1-|a|^2)$.\\
En d\'eduire que si $|a|<1$, alors $|z|=1$ ssi $f(a,z)=|\frac{z-a}{\overline{a}z-1}|=1$.\\
Que se passe-t-il si $|z|<1$?
\section{Identit\'e de Lagrange et l'in\'egalit\'e de Cauchy-Schwarz}
\emph{Page 20, Complex Analysis, Eberhard Greitag \& Rolf Busam}

\end{document}
