\documentclass{article}
\usepackage{amsmath}
\usepackage{amssymb}
\usepackage{geometry}
\begin{document}
\renewcommand\partname{Partie}

\title{Colles MPSI, Lyc\'ee Janson de Sailly}
\author{\textsf{HUANG Yichao}}
\date{2011-2012}

\maketitle
\tableofcontents
\newpage

\part{Exercices par chapitre}

\section{Nombres complexes}
\subsection{Questions du cours}
Q: Racines $n$-i\`emes (Racines)\\
Soit $a\in\mathbb{C}$. Montrer que $a$ admet une racine $n$-i\`eme, i.e. il existe $z\in\mathbb{C}$ tel que $z^n=a$. Combien y en a-t-il?\\
On pourra commencer par regarder les racines $n$-i\`emes de l'unit\'e.\\
Dessiner les racines $n$-i\`emes de l'unit\'e. Conjucturer la somme de toutes les racines $n$-i\`emes de l'unit\'e. Prouver ce r\'esultat.\\
Attention: dans toutes les questions il faut discuter selon la valeur de $n$!
\begin{center}
|||||
\end{center}
Q: In\'egalit\'e triangulaire (Modules)\\
Soient $z,z'\in\mathbb{C}$. Montrer que $|z+z'|\leq|z|+|z'|$. En d\'eduire que $||z|-|z'||\leq|z-z'|$. Quand est-ce qu'on a l'\'egalit\'e?
\begin{center}
|||||
\end{center}
Q: Formules trigonom\'etriques\\
D\'evelopper $\sin(\alpha+\beta)$ et $\cos(\alpha+\beta)$ o\`u $\alpha$ et $\beta$ sont deux nombres r\'eels.
\subsection{Exercices}
Ex: Somme de $\cos$ (Racines)\\
Calculer $\sum\limits_{k=0}^{n}\cos(k\theta)$, o\`u $\theta\in\mathbb{R}$.\\
Indication: Utiliser la technique de l'angle moiti\'e.
\begin{center}
|||||
\end{center}
Ex: Droite r\'eelle (Conjugaison)\\
Soient $a,b\in\mathbb{U}$, $z\in\mathbb{C}$. Notons $u=\frac{z+ab\overline{z}-a-b}{a-b}$. Montrer que $u^2\in\mathbb{R}$.
\begin{center}
|||||
\end{center}
Ex: In\'egalit\'e de Bell\\
Montrer qu'il existe des points $a$, $a'$, $b$, $b'$ sur la sph\'ere unit\'e de l'espace tels que $(<a,b>+<b,a'>+<a',b'>-<b',a>)\geq 2$ (on confond ici les points avec leurs coordon\'ees).
\begin{center}
|||||
\end{center}
Ex: Demi-plan complexe (Modules)\\
On d\'efinit $\mathbb{H}=\{z\in\mathbb{C}\mid Im~z>0\}$ le demi-plan du plan complexe.\\
Montrer que $z\in\mathbb{H}$ ssi $-\frac{1}{z}\in\mathbb{H}$.\\
On se donne $z,a$ deux nombres complexes.\\
Montrer que $|1-z\overline{a}|^2-|z-a|^2=(1-|z|^2)(1-|a|^2)$.\\
En d\'eduire que si $|a|<1$, alors $|z|=1$ ssi $f(a,z)=|\frac{z-a}{\overline{a}z-1}|=1$.\\
Que se passe-t-il si $|z|<1$?
\begin{center}
|||||
\end{center}
Ex: Polyn\^ome de Tchebychev\\
...
\begin{center}
|||||
\end{center}
Ex: Formule de De Moivre (Racine)\\
Montrer qu'on a l'identit\'e suivante:\\
$(\cos\phi+i\sin\phi)^n=\cos n\phi+i\sin n\phi$.
\begin{center}
|||||
\end{center}
Ex: Th\'eor\`eme de Napol\'eon\\
...
\begin{center}
|||||
\end{center}
Ex: Triangle \'equilat\'eral (Racines)\\
Montrer qu'il n'existe pas de triangle \'equilat\'eral \`a coordon\'ees enti\`eres (non r\'eduit \`a un point) dans $\mathbb{R}^2$.\\
(Indication: $a^2+b^2+c^2-(ab+bc+ca)=0$)
\begin{center}
|||||
\end{center}
Ex: Racine de l'unit\'e (Racines)\\
Calculer:\\
1) $(\frac{1+i}{\sqrt{2}})^n$, $n\in\mathbb{Z}$;\\
2) $\sum\limits_{k=0}^{7}(\frac{1-i}{\sqrt{2}})^k$.
\begin{center}
|||||
\end{center}
Ex: Polyn\^ome et racines $n$-i\`emes de l'unit\'e\\
Soit $n\in\mathbb{N^*}$. Soit $p$ un entier.\\
1) Calculer $\sum\limits_{\omega\in\mathbb{U}_n}\omega^p$.\\
2) Soit $P$ un polyn\^ome de degr\'e inf\'erieur ou \'egal \`a $n-1$. Montrer que $a_k=\sum\limits_{\omega\in\mathbb{U}_n}\frac{P(x)}{nx^k}$.
\begin{center}
|||||
\end{center}
Ex: Arithm\'etique? (Modules)\\
Montrer que si $a$, $b$ peuvent s'\'ecrire comme la somme de deux carr\'es dans $\mathbb{N}$, alors leur produit $ab$ l'est aussi.\\
Soient $a,b,c,d\in\mathbb{Z}$ tels que $(a+bi)(c+di)=1$. Exhiber tous les cas possibles.
\begin{center}
|||||
\end{center}
Ex: Noyau du F\'ejer\\
On introduit le noyau de Dirichlet d'ordre $n$ la fonction sur $\mathbb{R}$ d\'efini par: $D_n(x)=\sum\limits_{k=-n}^{n}e^{ikx}$.\\
1) Montrer que $D_n(x)\in\mathbb{R}$.\\
2) Montrer que $D_n(x)=1+2\sum\limits_{k=1}^{n}\cos(kx)$.\\
3) Montrer que $D_n$ est $2\pi$-p\'eriodique, i.e. $D_n(x)=D_n(x+2\pi)$.\\
4) Simplifier cette somme. (En discutant selon la divisibilit\'e de $x$ par $2\pi$.)\\
On introduit le noyan de F\'ejer d'ordre $n$ la fonction sur $\mathbb{R}$ d\'efini par: $F(x)=\frac{1}{n}\sum\limits_{k=0}^{n-1}D_k(x)$.\\
5) Montrer que $F(x)=\frac{1}{2n\pi}(\frac{\sin nx/2}{\sin x/2})^2$.\\
On donne ensuite sans d\'emonstration quelques propri\'et\'es sympas de ce noyau, en particulier c'est une approximation de Dirac sur $[-\pi,\pi]$.
\begin{center}
|||||
\end{center}
Ex: Extrait de ``Un lemme de confinement'' (X-ENS) (Modules)\\
Soient $\epsilon_{i}\in\{-1,1\}$ et $a_{i}$ des nombres complexes de module $1$.\\
Montrer:\\
1) On peut choisir $\epsilon_1$ et $\epsilon_2$ de sorte que $|\epsilon_1 a_1+\epsilon_2 a_2|\leq\sqrt{3}$. Exhiber un cas o\`u l'\'egalit\'e est atteinte.\\
2) Montrer la m\^eme assertion pour trois nombres.

\section{Fonctions usuelles}
\subsection{Questions du cours}
Q: $\arccos$ et $\arcsin$\\
Domaines de d\'efinitions, d\'erivations (attention aux domaines de d\'erivabilit\'e), graphes.\\
D\'emontrer l'identit\'e suivante: Pour tout $x\in[-1,1]$, $\cos\arcsin x=\sin\arccos x=\sqrt{1-x^2}$.
\begin{center}
|||||
\end{center}
\subsection{Exercices}
Ex: D\'erivation\\
Calculer $\sum\limits_{k=0}^{n}k\cos(k\theta)$ o\`u $\theta\in\mathbb{R}$.
\begin{center}
|||||
\end{center}
Ex: $\arctan$\\
R\'esoudre l'\'equation: $\arctan{2x}+\arctan{3x}=\frac{\pi}{4}$ d'inconnue $x\in\mathbb{R}$.\\
(R\'eponse: $x=\frac{1}{6}$. Penser \`a bien pr\'eciser l'injectivit\'e de $\arctan$ sur un intervalle appropri\'e.)
\begin{center}
|||||
\end{center}
Ex: Ruse de substitution\\
Soient $a,b\in\mathbb{R}$. Il existe alors $u,v\in\mathbb{R}$ tels que pour tout $\theta\in\mathbb{R}$,\\
$a\cos\theta+b\sin\theta=\sqrt{a^2+b^2}\cos(\theta+u)=\sqrt{a^2+b^2}\sin(\theta+v)$.
\begin{center}
|||||
\end{center}
Ex: Approximeation de $\frac{\pi}{4}$\\
Montrer que $4\arctan\frac{1}{5}-\arctan\frac{1}{239}=\frac{\pi}{4}$.
\begin{center}
|||||
\end{center}
Ex: Une somme t\'el\'escopique d\'eguis\'ee\\
Soit $x\in\mathbb{R*}$.\\
Montrer que $\tanh{x}=2\coth{x}-\coth{x}$. Conjecturer la somme de la s\'erie de terme $\frac{1}{2^n}\tanh\frac{x}{2^n}$.\\
Calculer la somme de $\frac{1}{\sinh(2^nx)}$.\\
(Ind: $\frac{1}{\sinh x}=\coth\frac{x}{2}-\coth x$.)
\begin{center}
|||||
\end{center}
Ex: Somme de $\arctan$\\
Montrer que $\arctan\frac{1}{3}+\arctan\frac{1}{2}+\arctan{1}=\frac{\pi}{2}$.
\begin{center}
|||||
\end{center}
Ex: La fonction $\sin$ n'est pas rationnelle (X-ENS)\\
Montrer que la fonction $\sin$ n'est pas rationnelle sur aucun intervalle r\'eel $[a,b]$.\\
On admettra le r\'esultat suivant: une fonction rationnelle n'a qu'un nombre fini de z\'ero sur un intervalle non vide $]a,b[$.\\
On rappelle que le degr\'e d'une fonction rationnelle $\frac{P}{Q}$ est d\'efini par $deg(P)-deg(Q)$ si $P\neq 0$ et $-\infty$ sinon. Cette d\'efinition ne d\'epend pas de la repr\'esentation choisie.

\section{\'Equations diff\'erentielles}


\section{Groupes, anneaux, corps}
Ex: Graphe de Cayley (cf. partie II).\\
Ex: Th\'eor\`eme de Cayley.\\
Ex: Lemme du serpent.\\
Ex: Anneau int\`egre fini.\\
Ex: Factorisation \`a travers le noyau.
\section{\'Equations diff\'erentielles lin\'eaires}
Ex: Probl\`eme de raccordement.
\section{G\'eom\'etrie \'el\'ementaire du plan et de l'espace}
Ex:
\section{Courbes param\'etr\'ees}
Ex:
\section{Coniques}
Ex:
\section{Ensembles des nombres}
Ex: $\sqrt{2}$ (ainsi que $2^{1/n}$ pour $n$ entier naturel $\geq 2$) n'est pas un nombre rationnel.\\
Ex: R\'ecurrence (exemples de fausses r\'ecurrences).\\
Ex: Plongement de $\mathbb{N}$ dans $\mathbb{Q}$.
\section{Suites num\'eriques}
Ex:
\section{Fonctions d'une variable r\'eelle \`a valeurs r\'eelles}
Ex:
\section{Arithm\'etique dans $\mathbb{Z}$}
Ex: \'Etude sur la fonction de M\"obius. Formule d'inversion de M\"obius.\\
On rappelle que chaque entier naturel $n$ admet une unique d\'ecomposition en produit de nombres premiers positifs, i.e. $n=p_1^{v_1}\dots p_r^{v_r}$, o\`u les $v_i$ sont appel\'es valuations en $p_i$.\\
La fonction de M\"obius $\mu$ est d\'efinie de la fa\c con suivante:\\
i) Si $r=0$, ou si les $v_i=1$ pour tout $i$ (ou en anglaise, $n$ est \emph{square-free}), alors $\mu(n)=(-1)^r$.\\
ii) Sinon, $\mu(n)=0$.\\
Le but de cet exercice est de d\'emontrer la formule suivante, dite formule d'inversion de M\"obius:\\
Soient $f$ et $g$ deux fonctions d\'efinies sur $\mathbb{N}$ \`a valeurs dans $\mathbb{R}$ (ou plus g\'en\'eralement, dans un groupe ab\'elien quelconque).\\
Si $\forall n\in\mathbb{N}, \sum\limits_{d|n}f(d)=g(n)$, alors $\forall n\in\mathbb{N}, \sum\limits_{d|n}\mu(n/d)g(d)=f(n)$.
\begin{center}
|||||
\end{center}
Ex: Applications du petit th\'eor\`eme de Fermat.\\
Soient $a$, $b$ deux entiers naturels et $p$ un nombre premier positif. Montrer que:\\
Si $a^p\equiv b^p$ (mod $p$), alors $a^p\equiv b^p$ (mod $p^2$).
\begin{center}
|||||
\end{center}
Ex: \'Ecriture en base $2$.\\
(1995 nansilafu) Notons $\lambda$ la fonction qui \`a un entier naturel $n$ associe le nombre de $1$ qui appara\^it dans son \'ecriture en base $2$. Alors $2^{n-\lambda(n)}|n!$.

\part{Exercices ``Originaux''}
\section{Graphe de Cayley}
Montrer \`a l'aide de deux figures qu'un groupe \`a quatre \'el\'ements est soit isomorphe \`a $\mathbb{Z}/2\mathbb{Z}\times \mathbb{Z}/2\mathbb{Z}$, soit \`a $\mathbb{Z}/4\mathbb{Z}$.\\
Int\'er\^et: Une repr\'esentation graphique du groupe. Classification.

\section{Groupe de Rubik}
On se donne un Rubik de taille deux. \'Etudions les actions, reconna\^itre un groupe familier.\\
Int\'er\^et: Initiation \`a la notion de l'action.

\section{\'Ellipse de Steiner}

\section{Polyn\^ome de Hurwitz}

\section{Th\'eor\`eme de Ping-Pong}

\section{Polyn\^ome de Laguerre}

\section{D\'eterminant de Hankel}
Suite de Fibonacci.\\
Int\'er\^et: Rien \`a part le plaisir d'une amuse-gueule...\\
Nombre de Catalan\footnote{http://www.emis.de/journals/JIS/VOL4/LAYMAN/hankel.pdf}.\\
Int\'er\^et: \`A voir.

\section{In\'egalit\'e de Hadamard}
Int\'er\^et: Interpr\'etation g\'eom\'etrique du d\'eterminant.

\section{R\'esultant}

\section{Th\'eor\`eme de Cayley}
Plongement d'un groupe dans un groupe sym\'etrique.

\section{Probl\`eme de distance de Erd\H os}
Ceci est une question ouverte.\\
Soit $N$ un entier naturel. Estimer $\inf\#\{d(x_i,x_j), i,j\in\{1,2,\dots N\}\}$, o\`u $x_i$ est un point dans le plan euclidien.\\
On demande d'\'etablir quelques r\'esultats pr\'eliminaires:\\
Notons $G_m(n)$ cette borne inf pour $n$ dans un espace euclidien de dimension $m$.\\
Montrer que $G$ est une fonction croissante en $m$ et en $n$.\\
Calculer $G_1{n}$ pour tout $n$ entier naturel.\\
On s'int\'eresse au cas $m=2$. Donner $G_2(n)$ pour $n=1,2,3,4,5,6,7,8,9,10$.\\
Montrer que $G_2(n)\geq n^{1/2}$.

\section{Fractions continues et l'\'equation de Pell-Fermat}

\section{Lemme de (s)tresses}

\section{Infinitude du nombres premiers et topologie g\'en\'erale}

\part{Mn\'emotechniques...}
\section{Th\'eor\`eme de Cantor}
On a un plongement naturel de $E$ dans $\mathcal{P}(E)$. Maintenant on consid\'ere un ensemble quelconque autre que les images, sa pr\'eimage n'est pas dans cet ensemble car l'image de cette pr\'eimage est d\'ej\`a d\'efinie. D'o\`u la consid\'eration: $A=\{a/a\notin f(a)\}$.


\part{Bizutages}
\section{Travail \'equipe}
Je peux leur faire travailler en \'equipe: la r\'esolution de l'un n\'ecessite le r\'esultat de l'autre.


\end{document}
