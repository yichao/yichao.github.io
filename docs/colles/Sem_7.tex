\documentclass{article}
\usepackage{amsmath}
\usepackage{amssymb}
\usepackage{geometry}
\begin{document}

\title{\textsc{Coniques}\footnote{Un bouquin qui n'utilise ``aucune'' \'equation pour d\'emontrer pas mal de propri\'et\'es sur les coniques: \textsc{A treatise on geometrical conics} [Arthur Cockshott, F.B. Walters]. http://ebook.lib.hku.hk/CADAL/B31395533/}}
\author{www.eleves.ens.fr/home/yhuang}
\date{24 Novembre, 2011}

\maketitle

\section{Visual Sieve}
On consid\`ere l'ensemble $C$ des droites passant par deux points entiers $(-a,a^2),(b,b^2)$ sur la parabole $y=x^2$ avec $a,b\in\mathbb{N}-\{0,1\}$.\\
Montrer que $C$ \'evite tous les points $(0,p)$ avec $p$ premier et rencontre tous les points $(0,q)$ avec $q$ non premier \`a deux exceptions $(0,0)$ et $(0,1)$ pr\`es.\\
Cette m\'ethode est due \`a Yuri Matiyasevich et Boris Stechkin.

\section{Coffin Problem\protect\footnote{Pour une br\`eve histoire, consulter http://www.tanyakhovanova.com/coffins.html}}
Il n'existe pas d'ensemble form\'e d'un nombre infini de points dans le plan tels que la distance entre deux points quelconques soit un entier.

\section{Triangle sur une hyperbole}
Montrer que l'orthocentre d'un triangle dont les sommets appartiennent \`a l'hyperbole $xy=1$ l'appartient. Montrer que le cercle circonscrit de ce triangle va intersecter l'hyperbole en un point sym\'etrique de l'orthocentre par rapport \`a l'origine.

\section{Exercice Classique}
Soit $H$ une hyperbole dont les asymptote se coupent en $O$. Soit $M$ un point sur $H$, alors la tangente en $M$ coupe les asymptotes en deux points $P,Q$. Montrer que l'aire de $MPQ$ ne d\'epend pas de $M$.

\section{Billards Elliptiques}
On se donne une table de billard elliptique et une balle situ\'ee en un foyer. Apr\`es un rebond, la balle passe par l'autre foyer. Essayer de retrouver ce r\'esultat \`a l'aide du principe de Fermat.

\section{Suite de l'exercice pr\'ec\'edent: billards elliptiques\protect\footnote{Merci \`a Weikun pour m'avoir communiqu\'e cet exercice.}}
Fixons une trajectoire dans un billard elliptique.\\
Cas 1): La trajectoire ne passe jamais par le segment reliant les deux foyers. Alors il existe une ellipse (de m\^eme foyers) telle qu'elle soit tangente aux tous les segments entre deux rebonds, et qui ne d\'epend que de la trajectoire. Pour bien d\'emarrer, on pourra se rappeler du th\'eor\`eme de Poncelet.\\
Cas 2): La trajectoire passe toujours par le segment reliant les deux foyers. Alors il existe une hyperbole (de m\^eme foyers) telle qu'elle soit tangente aux tous les segments entre deux rebonds, et qui ne d\'epend que de la trajectoire.\\
On pourra, pour simplifier, faire d'abord le cas d'une ellipse d'excentricit\'e $0$, i.e. un cercle.

\section{Th\'eor\`eme de Pascal}

\section{Th\'eor\`eme du papillon g\'en\'eralis\'e}

\section{Exercice Calculatoire}
R\'eduire la conique d'\'equation: $\cosh(t)(x^2+y^2)-2\sinh(t)xy+e^t\cos(t)(x-y)+\frac{e^t}{4}=0$ avec $t\in\mathbb{R}$.

\section{Exercice ``Hors-Programme''\protect\footnote{Cet exercice n'est pas difficile du tout, mais il demande un peu de connaissance sur les applications affines.}}
Montrer que pour tout triangle, il existe une ellipse tantgente aux trois c\^ot\'es en leur milieu.

\section{Suite de l'exercice pr\'ec\'edent: polyn\^ome, racines et ellipse\protect\footnote{Merci \`a Miguel pour m'avoir communiqu\'e cet exercice.}}
Cas simple: prenons le polyn\^ome $P(X)=(X^2+X+1)(X-\frac{3}{2})=X^3-\frac{1}{2}X^2-\frac{1}{2}X-\frac{3}{2}$. Trouver le triangle $T$ de sommets les racines de $P$ dans le plan complexe. Trouver les racines de $P'$ dans le plan complexe. Trouver une ellipse $E$ ayant pour foyers les racines de sa d\'eriv\'ee et tangente aux trois c\^ot\'es du triangle $T$. Remarques sur $E\cap T$?\\
Cas g\'en\'eral: Soit $P$ un polyn\^ome s\'eparable (i.e. ses racines sont deux \`a deux distinctes) de degr\'e $2$. Refaire l'exercice pr\'ec\'edent.\\
On pourra admettre le th\'eor\`eme fondamental de l'alg\`ebre, m\^eme si le cas de degr\'e $2$ (donc a fortiori de degr\'e $3$) a d\'ej\`a \'et\'e vu dans le cours.

\end{document}
