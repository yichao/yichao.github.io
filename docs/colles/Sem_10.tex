\documentclass{article}
\usepackage{amsmath}
\usepackage{amssymb}
\usepackage{geometry}
\begin{document}

\setcounter{section}{10}

\title{\textsc{Nombres R\'eels}}
\author{www.eleves.ens.fr/home/yhuang}
\date{\textsf{Draft!}}
\maketitle

\subsection{Quelques in\'egalit\'es classiques}
\subsubsection{In\'egalit\'e de Schur}
Soient $a,b,c,k$ r\'eels positifs. Montrer que $a^k(a-b)(a-c)+b^k(b-c)(b-a)+c^k(c-a)(c-b)\geq 0$.
\subsubsection{In\'egalit\'e de Tchebychev}
1) Soient $a_1\geq a_2\geq\dots\geq a_n\geq0$, $b_1\geq b_2\geq\dots\geq b_n\geq0$. Alors $\frac{1}{n}\sum\limits_{k=1}^{n}(a_kb_k)\geq\frac{\sum\limits_{k=1}^{n}a_k}{n}\frac{\sum\limits_{k=1}^{n}b_k}{n}$.\\
2) Soient $a_1\geq a_2\geq\dots\geq a_n\geq0$, $0\leq b_1\leq b_2\leq\dots\leq b_n$.Alors $\frac{1}{n}\sum\limits_{k=1}^{n}(a_kb_k)\leq\frac{\sum\limits_{k=1}^{n}a_k}{n}\frac{\sum\limits_{k=1}^{n}b_k}{n}$.
\subsubsection{Exemples}
1) Montrer que $\frac{1}{a}+\frac{1}{b}+\frac{1}{c}\leq\frac{a^8+b^8+c^8}{a^3b^3c^3}$.\\
2) Montrer que si $a,b,c$ sont les longueurs des c\^ot\'es d'un triangle, alors $\dots$.

\subsection{Quelques exercices sur la fonction partie enti\`ere}
\subsubsection{Une \'egalit\'e}
$\forall n\in\mathbb{N^*}$, $E(\sqrt{n})+E(\sqrt{n+1})=E(\sqrt{4n+2})$.
\subsubsection{Une autre \'egalit\'e}
$\forall n\in\mathbb{N^*}$, $\sum\limits_{k=0}^{n-1}E(x+\frac{k}{n})=E(nx)$.

\subsection{$(1+\sqrt{2})^n$}
Montrer qu'il existe un couple unique de suites \`a valeurs enti\`eres $(a_n)$ et $b_n$ telles que:\\
1) $(1+\sqrt{2})^n=a_n+b_n\sqrt{2}$.\\
2) Montrer que $|a_n^2-2b_n^2|=1$.

\subsection{Construction de $\mathbb{R}$ \`a partir de $\mathbb{Q}$}

\subsection{Parties enti\`eres et divisibilit\'e}
Montrer que $\frac{(2m)!(2n)!}{(m+n)!m!n!}$ est un entier.

\subsection{Exemple de partie dense dans $\mathbb{R}$}
Montrer que l'ensemble des nombres 2-adiques est dense dans $\mathbb{R}$.

\subsection{Approximation d'un irrationnel par un rationnel}
On note, pour un nombre r\'eel $\alpha$, $||\alpha||$ la distance entre $\alpha$ et l'ensemble des entiers. Concr\`etement, $||\alpha||=\min\{\alpha-E(\alpha), E(\alpha)+1-\alpha\}$.\\
1) Montrer que $\forall z\in\mathbb{Z}, ||\alpha+z||=||\alpha||$.\\
Soient $\alpha\in\mathbb{R}$, $Q\in\mathbb{R}$.\\
2) Montrer que $\exists q\in\mathbb{Z}, 0<q<Q$, $||q\alpha||<Q^{-1}$.

\subsection{Compl\'ements}

\subsubsection{Valuations $p$-adiques et ultram\'etrique}
On dit qu'une fonction de $\mathbb{Q}\to\mathbb{R^+}$ est une fonction valeur absolue si:\\
i) \\
ii) \\
iii) \\
iv) , et si on plus \dots, on dit que cette fonction valeur absolue est ultram\'etrique.\\
Soit $p$ un nombre premier. On d\'efinit, pour tout $z\in\mathbb{Z}$, la valuation $p$-adique $\sup\{i\in\mathbb{N},p^i|z\}$. Cet entier est alors unique et on le note $val_p(z)$.\\
On \'etend ensuite cette d\'efinition aux nombres rationnels en d\'efinissante, pour un nombre rationnel $\frac{u}{v}$, $val_p(\frac{u}{v})=val_p(u)-val_p(v)$.\\
Montrer que $val_p$ est une fonction valeur absolue ultram\'etrique.

\subsubsection{Th\'eor\`eme d'Ostrowski}
On d\'efinit sur l'ensemble des fonctions valeur absolue sur $\mathbb{Q}$ la relation d'\'equivalence suivante: $|.|\sim||.||$ ssi $\exists c\in\mathbb{R^*}$, $\frac{1}{c}|.|\leq||.||\leq c|.|$ (et on admet que c'est une relation d'\'equivalence).\\
Monter que toute fonction valeur absolue sur $\mathbb{R}$ non triviale est soit \'equivalente \`a la valeur absolue usuelle, soit \`a une valeur absolue $p$-adique ci-dessus.

\end{document}
