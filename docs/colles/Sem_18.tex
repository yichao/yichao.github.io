\documentclass{article}
\usepackage{amsmath}
\usepackage{amssymb}
\usepackage{geometry}
\begin{document}

\setcounter{section}{18}

\title{\textsc{Entiers naturels}}
\author{www.eleves.ens.fr/home/yhuang}
\date{}
\maketitle

\noindent{La notation $C_n^k$ d\'esigne le nombre de sous-ensembles \`a $k$ \'el\'ements dans un ensemble \`a $n$ \'el\'ements.}

\subsection{$n=2^m(2k+1)$}
Tout entier naturel non nul $n$ s'\'ecrit de mani\`ere unique sous la forme $2^{m}(2k+1)$ avec $(m,k)\in\mathbb{N}\times\mathbb{Z}$.\\
On pourra faire une r\'ecurrence forte sur $|n|$, ou bien consid\'erer l'ensemble $\{m\in\mathbb{N}, 2^m|n\}$.

\subsection{$\sum\limits_{k}n_k k!$}
Soit $p\in\mathbb{N}$. Soit $0\leq n\leq (p+1)!-1$ un entier naturel. Montrer qu'il existe une unique suite $(n_k)_{0\leq k\leq p}$ avec $\forall k, 0\leq n_k\leq k$ telle que $n=\sum\limits_{k}n_k k!$.

\subsection{Une somme binomiale}
Soit $n\in\mathbb{N}$. Calculer $\sum\limits_{k\in\mathbb{N}, 0\leq 2k\leq n}C_n^k$. On pourra penser \`a former son ``compl\'ementaire''.

\subsection{Une identit\'e}
Soit $(n,p,q)\in\mathbb{N}^3$ avec $n\leq p+q$. En regardant $(1+x)^p(1+x)^q$, montrer que $C_{p+q}^n=\sum\limits_{0\leq k\leq n}C_p^k C_q^{n-k}$.\\
Proposer une interpr\'etation combinatoire.

\subsection{Un calcul}
Soit $n\in\mathbb{N}$. Calculer $\sum\limits_{0\leq k\leq n}(-1)^kC_{2n+1}^k$.

\subsection{Formule d'inversion de Pascal}
Soient $(u_n)$, $(v_n)$ deux suites telles que $\forall n\in\mathbb{N}, v_n=\sum\limits_{0\leq k\leq n}C_n^k u_k$.\\
Montrer que $\forall n\in\mathbb{N}, u_n=\sum\limits_{0\leq k \leq n}(-1)^k C_n^k v_k$.

\subsection{Une \'equation binomiale}
Consid\'erons l'\'equation diophantienne $C_{n+1}^{k+1}=C_{n}^{k+2}$ d'inconnues $(k,n)\in\mathbb{N}^2$. Montrer que si $F_i$ est le $i$-i\`eme nombre de Fibonacci, alors le couple $(n=F_{2i+2}F_{2i+3}-1, k=F_{2i}F_{2i+3}-1)$ est une solution d'\'equation pour tout $i\in\mathbb{N}$.\footnote{Ceci permet de montrer certaines propri\'et\'es sur le triangle de Pascal, cf. Wikip\'edia.}\\


\end{document}