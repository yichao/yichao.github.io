\documentclass{article}
\usepackage{amsmath}
\usepackage{amssymb}
\usepackage{geometry}
\begin{document}

\title{\textsc{Nombres Complexes \& Fonctions Usuelles}}
\author{www.eleves.ens.fr/home/yhuang}
\date{29 Septembre, 2011}

\maketitle

\section{Somme de $\cos$}
Calculer $\sum\limits_{k=0}^{n}\cos(k\theta)$, o\`u $\theta\in\mathbb{R}$.\\
Indication: Utiliser la technique de l'angle moiti\'e.

\section{Demi-plan complexe}
On d\'efinit $\mathbb{H}=\{z\in\mathbb{C}\mid Im~z>0\}$ le demi-plan du plan complexe.\\
Montrer que $z\in\mathbb{H}$ ssi $-\frac{1}{z}\in\mathbb{H}$.\\
On se donne $z,a$ deux nombres complexes.\\
Montrer que $|1-z\overline{a}|^2-|z-a|^2=(1-|z|^2)(1-|a|^2)$.\\
En d\'eduire que si $|a|<1$, alors $|z|=1$ ssi $f(a,z)=|\frac{z-a}{\overline{a}z-1}|=1$.\\
Que se passe-t-il si $|z|<1$?

\section{Formule de De Moivre}
Montrer qu'on a l'identit\'e suivante:\\
$(\cos\phi+i\sin\phi)^n=\cos n\phi+i\sin n\phi$ avec $\phi\in\mathbb{R}$.\\
Solution: On remarque que $(e^{i\phi})^n=e^{in\phi}$, puis on identifie les parties r\'eelles et imaginaires.

\section{Th\'eor\`eme de Napol\'eon}

\section{Triangle \'equilat\'eral}
Montrer qu'il n'existe pas de triangle \'equilat\'eral \`a coordon\'ees enti\`eres (non r\'eduit \`a un point) dans $\mathbb{R}^2$.\\
(Indication: $a^2+b^2+c^2-(ab+bc+ca)=0$)\\
Solution: Cette indication n'est pas bonne. C'est plut\^ot une \'etape dans la d\'emonstration de cette indication qu'il faudrait utiliser.\\
En effet, quitte \`a r\'eorienter, on peut supposer que $\frac{a-c}{b-c}=j$ o\`u $j$ est une racine $3$-i\`eme primitive de l'unit\'e. On v\'erifie que $j=\frac{-1-\sqrt{3}}{2}$ ou $j=\frac{-1+\sqrt{3}}{2}$. Alors en identifiant les parties r\'eelles et les parties imaginaires et en utilisant l'irrationalit\'e de $\sqrt{3}$, on obtient le r\'esultat.

\section{Polyn\^ome et racines $n$-i\`emes de l'unit\'e}
Soit $n\in\mathbb{N^*}$. Soit $p$ un entier.\\
1) Calculer $\sum\limits_{\omega\in\mathbb{U}_n}\omega^p$.\\
2) Soit $P$ un polyn\^ome de degr\'e inf\'erieur ou \'egal \`a $n-1$. Montrer que $a_k=\sum\limits_{\omega\in\mathbb{U}_n}\frac{P(\omega)}{n\omega^k}$.\\
Solution: C'est un cas explicite de l'interpolation de Lagrange...\\
Il s'agit dans la question 1) d'une somme g\'eom\'etrique.\\
Dans 2), il s'agit d'une somme double: on fait un Fubini, et on voit que tous les termes, \`a part celui de $x^k$, sont annul\'es par sym\'etrie. Finalement on retrouve la valeur de $a_k$.

\section{Arithm\'etique?}
Montrer que si $a$, $b$ peuvent s'\'ecrire comme la somme de deux carr\'es dans $\mathbb{N}$, alors leur produit $ab$ l'est aussi.\\
Soient $a,b,c,d\in\mathbb{Z}$ tels que $(a+bi)(c+di)=1$. Exhiber tous les cas possibles.

\section{Noyau du F\'ejer}
On introduit le noyau de Dirichlet d'ordre $n$ la fonction sur $\mathbb{R}$ d\'efini par: $D_n(x)=\sum\limits_{k=-n}^{n}e^{ikx}$.\\
1) Montrer que $D_n(x)\in\mathbb{R}$.\\
2) Montrer que $D_n(x)=1+2\sum\limits_{k=1}^{n}\cos(kx)$.\\
3) Montrer que $D_n$ est $2\pi$-p\'eriodique, i.e. $D_n(x)=D_n(x+2\pi)$.\\
4) Simplifier cette somme. (En discutant selon la divisibilit\'e de $x$ par $2\pi$.)\\
On introduit le noyan de F\'ejer d'ordre $n$ la fonction sur $\mathbb{R}$ d\'efini par: $F(x)=\frac{1}{n}\sum\limits_{k=0}^{n-1}D_k(x)$.\\
5) Montrer que $F(x)=\frac{1}{2n\pi}(\frac{\sin nx/2}{\sin x/2})^2$.\\
On donne ensuite sans d\'emonstration quelques propri\'et\'es sympas de ce noyau, en particulier c'est une approximation de Dirac sur $[-\pi,\pi]$.

\section{Extrait de ``Un lemme de confinement'' (X-ENS)}
Soient $\epsilon_{i}\in\{-1,1\}$ et $a_{i}$ des nombres complexes de module $1$.\\
Montrer:\\
1) On peut choisir $\epsilon_1$ et $\epsilon_2$ de sorte que $|\epsilon_1 a_1+\epsilon_2 a_2|\leq\sqrt{3}$. Exhiber un cas o\`u l'\'egalit\'e est atteinte.\\
2) Montrer la m\^eme assertion pour trois nombres.

\section{Arctan}
R\'esoudre l'\'equation: $\arctan{2x}+\arctan{3x}=\frac{\pi}{4}$ d'inconnue $x\in\mathbb{R}$.\\
(R\'eponse: $x=\frac{1}{6}$. Penser \`a bien pr\'eciser l'injectivit\'e de $\arctan$ sur un intervalle appropri\'e.)

\section{Ruse de substitution}
Soient $a,b\in\mathbb{R}$. Il existe alors $u,v\in\mathbb{R}$ tels que pour tout $\theta\in\mathbb{R}$,\\
$a\cos\theta+b\sin\theta=\sqrt{a^2+b^2}\cos(\theta+u)=\sqrt{a^2+b^2}\sin(\theta+v)$.

\section{Approximeation de $\frac{\pi}{4}$}
Montrer que $4\arctan\frac{1}{5}-\arctan\frac{1}{239}=\frac{\pi}{4}$.

\section{Une somme t\'el\'escopique d\'eguis\'ee}
Soit $x\in\mathbb{R*}$.\\
Montrer que $\tanh{x}=2\coth{x}-\coth{x}$. Conjecturer la somme de la s\'erie de terme $\frac{1}{2^n}\tanh\frac{x}{2^n}$.\\
Calculer la somme de $\frac{1}{\sinh(2^nx)}$.\\
(Ind: $\frac{1}{\sinh x}=\coth\frac{x}{2}-\coth x$.)

\section{Somme de $\arctan$}
Montrer que $\arctan\frac{1}{3}+\arctan\frac{1}{2}+\arctan{1}=\frac{\pi}{2}$.

\section{La fonction $\sin$ n'est pas rationnelle (X-ENS)}
Montrer que la fonction $\sin$ n'est pas rationnelle sur aucun intervalle r\'eel $[a,b]$.\\
On admettra le r\'esultat suivant: une fonction rationnelle n'a qu'un nombre fini de z\'ero sur un intervalle non vide $]a,b[$.\\
On rappelle que le degr\'e d'une fonction rationnelle $\frac{P}{Q}$ est d\'efini par $deg(P)-deg(Q)$ si $P\neq 0$ et $-\infty$ sinon. Cette d\'efinition ne d\'epend pas de la repr\'esentation choisie.

\end{document}
