\documentclass{article}
\usepackage{amsmath}
\usepackage{amssymb}
\usepackage{geometry}
\begin{document}

\setcounter{section}{16}

\title{\textsc{Continuit\'e, Uniforme continu\'e}}
\author{www.eleves.ens.fr/home/yhuang}
\date{}
\maketitle

\subsection{Vrai ou faux}
Les fonctions sont-elles uniform\'ement continues?\\
1) $x\mapsto\sqrt{x}$ sur $\mathbb{R}^+$.\\
2) $x\mapsto\ln x$ sur $\mathbb{R}^{+*}$.

\subsection{Exo Classique}
Soit $f:[0,1]\to[0,1]$ une fonction continue. Montrer que la suite d\'efinie par $x_{n+1}=f(x_n)$ avec $x_0\in[0,1]$ converge ssi $\lim\limits_{n\in\mathbb{N}}(x_n-x_{n+1})=0$.\\
On pourra montrer d'abord qu'une suite born\'ee non convergente admet au moins deux valeurs d'adh\'erences.

\subsection{Fonction uniform\'ement continue et fonction affine}
Montrer qu'une fonction $f:\mathbb{R}^+\to\mathbb{R}$ est affinement born\'ee.

\subsection{Caract\'erisation s\'equentielle}
L'espace consid\'er\'e ici est $\mathbb{R}$.\\
1) Montrer qu'une fonction $f:\mathbb{R}\to\mathbb{R}$ est uniform\'ement continue ssi pour tout couple de suites $((x_n),(y_n))_{n\in\mathbb{N}}$ tel que $\lim\limits_{n\to\infty}|x_n-y_n|=0$, on a $\lim\limits_{n\to\infty}|f(x_n)-f(y_n)|=0$.\\
2) Montrer que l'image d'une suite de Cauchy par une fonction uniform\'ement continue est une suite de Cauchy.

\subsection{``Recollement''}
Soit $f:\mathbb{R}^+\to\mathbb{R}$ une fonction continue admettant une limite \`a l'infini. Montrer que $f$ est uniform\'ement continue.

\subsection{Module de continuit\'e}

\end{document}