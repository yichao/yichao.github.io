\documentclass{article}
\usepackage{amsmath}
\usepackage{amssymb}
\usepackage{geometry}
\begin{document}

\title{\textsc{G\'eom\'etrie Plane \& G\'eom\'etrie Spatiale}}
\author{www.eleves.ens.fr/home/yhuang}
\date{20 Octobre, 2011}

\maketitle

\section{Autour du th\'eor\`eme de Ptol\'em\'ee}

\textbf{D\'emonstration 1}\\
Lemme: \textbf{La loi des sinus}\\
1) Rappeler la d\'efinition du d\'eterminant. L'interpr\'eter g\'eom\'etriquement.\\
2) En d\'eduire la loi des sinus: dans un triangle non r\'eduit $ABC$, on a toujours $\frac{|AB|}{\sin c}=\frac{|BC|}{\sin b}=\frac{|CA|}{\sin a}$.\\
On pourrait s'amuser \`a trouver d'autres preuves, sans doute plus simples. Consulter Wikip\'edia par exemple\dots\\
3) D\'emontrer le th\'eor\`eme.\\
\\
\textbf{Ind:}\\
Il ne reste qu'\`a exprimer les rapports entre les longeurs en terme du rapport entre les sinus. Puis en utilisant la proposition au tout d\'ebut, les sinus se simplifient.\\
\\
\textbf{D\'emonstration 2}\\
Lemme: La loi des cosinus\\
\\
\textbf{Ind:}\\
Exprimer les diagonales \`a partir de la loi de cosinus puis \'eliminer les termes en $\cos$.\\
\\
\textbf{D\'emonstration 3}\\
Lemme: Une identit\'e trigonom\'etrique\\
D\'emontrer d'abord le lemme suivant (qui porte lui aussi le nom du th\'eor\`eme de Ptol\'em\'ee\dots)\\
Soient $a,b,c$ trois r\'eels quelconques. Montrer que $\sin(a)\sin(b)+\sin(c)\sin(a+b+c)=\sin(a+c)\sin(b+c)$.\\
Maintenant, en exprimant par exemple $|AB|=2\sin(\alpha/2)$ avec $\alpha$ un angle bien choisi, d\'emontrer le th\'eor\`eme de Ptol\'em\'ee.\\
\\
\textbf{Ind:}\\
Pour la formule trigonom\'etrique, il n'y a pas de difficult\'e majeure: il suffit d'utiliser l'identit\'e bien connue: $2\sin(x)\sin(y)=\cos(x-y)-\cos(x+y)$. Il reste \`a exprimer les c\^ot\'es par les angles en utilisant le th\'eor\`eme de l'angle inscrit (ce qui fournit d'ailleurs une autre preuve de la loi des sinus).\\
\\
\textbf{D\'emonstration 4}\\
Une astuce: Les triangles semblables, transformations du plan\\
Cette preuve est astucieuse.\\
\\
\textbf{Ind:}\\
L'astuce consiste \`a choisir judicieusement un point sur un diagonal, disons un point $P$ sur $AC$, \`a ce que $\triangle DBC\sim\triangle ABP$ et $\triangle ABD\sim\triangle PBC$.\\
\\
\textbf{D\'emonstration 5}\\
Nombres complexes, coordon\'ees polaires\\
\\
Cela revient \`a exprimer l'une des d\'emonstration en termes de nombres complexes\dots\\
\\
\textbf{D\'emonstration 6}\\
Rapport anharmonique\\
\\
\textbf{D\'emonstration 7}\\
D\'emonstration par inversion\\
On fixe un point $A$ sur le plan et un nombre r\'eel positif $k$. On appelle l'inversion de p\^ole $A$ et de puissance $k$ l'application qui \`a chaque point $M$ autre que $A$ de l'espace associe l'unique point $M'$ tel que $det(\overrightarrow{AM},\overrightarrow{AM'})=0$ et $\overrightarrow{AM}.\overrightarrow{AM'}=k$.\\
0) Montrer que le point $M'$ \`a pour expression: $M'=A+\frac{k.\overrightarrow{AM}}{||\overrightarrow{AM}||^2}$.\\
1) On d\'emontre un r\'esultat pr\'eliminaire: si $A,B,C,D$ \'etaient cocycliques, l'images de $B$, $C$, $D$ par l'inversion de p\^ole $A$ et de puissance quelconque seraient align\'ees.\\
Pour s'en convaincre, on utilise par exemple les nombres complexes. On suppose, quitte \`a faire une homoth\'etie et une translation que $A=0$ et $(B,C,D)=(1+e^{i\theta_1},1+e^{i\theta_2},1+e^{i\theta_3})$.\\
Montrer alors que l'images de $B,C,D$ par l'inversion de p\^ole $A=0$ et de puissance $1$ et bien une droite.\\
2) Utiliser ce r\'esultat pour d\'emontrer le th\'eor\`eme de Ptol\'em\'ee.\\
On pourra, si on est perdu dans les calculs, utiliser l'identit\'e suivante: $P'Q'=\frac{r^2.PQ}{OP.OQ}$.\\
\\
\textbf{Ind:}\\
Pour 1), on peut d\'emontrer que les parties r\'eelles des images sont \'egales \`a $\frac{1}{2}$.\\
Pour 2), c'est un calcul en utilisant les relation sur les triangles semblables.\\
\\
\textbf{D\'emonstration 8}\\
Partant de l'\'egalit\'e $(a-b)(c-d)+(a-d)(b-c)=(a-c)(b-d)$ dans $\mathbb{C}$, on obtient \textbf{l'in\'egalit\'e de Ptol\'em\'ee}.\\
On obtient ensuite l'\'egalit\'e de Ptol\'em\'ee.\\
\\
\textbf{Ind:}\\
Il suffit d'utiliser l'in\'egalit\'e triangulaire.\\
En outre, on a l'\'egalit\'e dans l'in\'egalit\'e pr\'ec\'edente ssi $Arg\frac{a-b}{a-d}=-Arg\frac{c-d}{c-b}$ (i.e. $\angle A+\angle C=2\pi$), ce qui revient \`a dire que les points $A,B,C,D$ sont cocycliques.\\
\\
\textbf{D\'emonstration 9}\\
Inversion muette.\\
On se propose de d\'emontrer l'in\'egalit\'e de Ptol\'em\'ee en utilisant l'in\'egalit\'e triangulaire.\\
1) Soient $x,y$ trois vecteurs quelconques du plan. Montrer que $|\frac{x}{|x|^2}-\frac{y}{|y|^2}|=\frac{|x-y|}{|x|.|y|}$.\\
C'est en fait une inversion de p\^ole $0$ et de puissance $1$\dots\\
2) Appliquons l'in\'egalit\'e triangulaire pour les vecteurs $b=\overrightarrow{AB}$, $c=\overrightarrow{AC}$, $d=\overrightarrow{AD}$ et en d\'eduire l'in\'egalit\'e de Ptol\'em\'ee. Interpr\'eter ce r\'esultat g\'eom\'etriquement.\\
3) Discuter le cas d'\'egalit\'e et conclure.
\section{Corollaires \& Applications}
\textbf{Corollaire 1}\\
Cas o\`u $ABC$ est un triangle \'equilat\'eral\\
\textbf{Corollaire 2}\\
(Difficile) Montrer que pour tout entier $n$, on peut trouve $n$ points sur le plan euclidien tels que la distance entre n'importe quels deux points soit un entier.\\
\textbf{Corollaire 3}\\
(Difficile *2) \textbf{Th\'eor\`eme de Casey} \& \textbf{Th\'eor\`eme de Euler-Feuerbach}
\section{Notes Historiques}
Ptolom\'ee ($90?$ - $168$) \'etait un astronome et astrologue grec qui v\'ecut en \'Egypte.
\section{Bonus: Th\'eor\`eme de Sylvester-Gallai}
On se donne un ensemble fini de points dans le plan non tous align\'es. Montrer qu'il existe une droite passant par exactement deux de ces points.\\
On pourra penser \`a utiliser l'argument de descente infinie.\\
\\
\textbf{Solution (Kelly):}\\
On note notre ensemble fini $\mathcal{F}$.\\
On note $\mathcal{D}$ l'ensemble des droites passant par deux points dans $\mathcal{F}$. Alors $\mathcal{D}$ est fini par un argument combinatoire.\\
Par un argument combinatoire du m\^eme type, l'ensemble des distances non nulles d'un point dans $F$ \`a une droite dans $\mathcal{D}$ est fini. On peut donc choisir un couple $(P\in\mathcal{F},D\in\mathcal{D}$ tel que la distance entre eux soit minimale.\\
Montrons qu'on peut alors trouver un autre couple tel que la distance entre ses deux \'el\'ements soit plus petite que la distance entre $P$ et $D$, ce qui sera absurde.\\
Par hypoth\`ese, $D$ contient au moins $3$ points. Donc si on note $H$ le point obtenu en projectant $P$ orthogonalement sur $D$, sans perte de g\'en\'eralit\'e, on peut supposer que les points $A,B\in\mathbb{F}$ sont sur le m\^eme cot\'e, avec $|PA|<|PB|$. Alors on v\'erifie que $A$ et la droite passant pas $B$ et $P$ est un couple qui convient.

\section{G\'eom\'etrie Spatiale}
\textbf{Ex: Formule du double produit vectoriel}\\
\'Etablir la formule suivante dans l'espace: $\overrightarrow{u}\wedge(\overrightarrow{v}\wedge\overrightarrow{w})=(\overrightarrow{u}.\overrightarrow{w})\overrightarrow{v}-(\overrightarrow{u}.\overrightarrow{v})\overrightarrow{w}$, o\`u $\wedge$ d\'esigne le produit vectoriel dans l'espace.\\
En d\'eduire l'ensemble des solutions de l'\'equation $\overrightarrow{a}\wedge\overrightarrow{x}=\overrightarrow{b}$ avec $\overrightarrow{a}\neq 0$.\\
On pourra traiter d'abord le cas o\`u $\overrightarrow{a}$ et $\overrightarrow{b}$ sont orthogonaux.

\section{Autres}
\textbf{X 2007}\\
Soient $z_1, z_2, \dots, z_n$ les sommets d'un polygone dans le plan complexe. Exprimer son aire.\\
|\\
\textbf{Ex: X 2007}\\
\`A quelle condition n\'ecessaire et suffisante portant sur les trois r\'eels $\alpha$, $\beta$, $\gamma$ existe-t-il un triangle $ABC$ pour lequel les trois \'egalit\'es $\overrightarrow{AB}.\overrightarrow{AC}=\alpha$, $\overrightarrow{BA}.\overrightarrow{BC}=\beta$, et $\overrightarrow{CA}.\overrightarrow{CB}=\gamma$ soient v\'erifi\'ees?\\
\\
\textbf{Ind:}\\
Utiliser le fait que $2|\overrightarrow{AB}.\overrightarrow{AC}|=|\overrightarrow{AB}|^2+|\overrightarrow{AC}|^2-|\overrightarrow{BC}|^2$.

\end{document}
