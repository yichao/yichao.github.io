\documentclass{article}
\usepackage{amsmath}
\usepackage{amssymb}
\usepackage{geometry}
\begin{document}

\setcounter{section}{13}

\title{\textsc{Groupes, Anneaux, Corps: II}\protect\footnote{Les anneaux sont suppos\'es commutatifs unitaires sauf mention contraire.}}
\author{www.eleves.ens.fr/home/yhuang}
\date{}
\maketitle

\subsection{Exemples d'anneaux}
\subsubsection{Anneaux des polyn\^omes}
Montrer que l'anneaux des polyn\^omes \`a coefficients r\'eels $\mathbb{R}[X]$ (muni de l'addition et la multiplication usuelles) et un anneau.
\subsubsection{Entiers de Gauss\protect\footnote{Tir\'e de Neukirch, P1-P2.}}
Consid\'erons $\mathbb{Z}[i]$ form\'e de tous les nombres complexes de la forme $a+b.i$ avec $(a,b)\in\mathbb{Z}^2$. Montrer que munissant de l'addition et la multiplication usuelles, il s'agit d'un sous-anneau de $\mathbb{C}$.\\
D\'eterminer son groupe des unit\'es.\\
Montrer que c'est un anneau euclidien (donc factoriel par l'exercice \ref{euclidien}).\\
On d\'efinit la norme d'un \'el\'ement dans $\mathbb{Z}[i]$ comme la somme des carr\'es de ses deux coefficients. Montrer que la norme est une fonction multiplicative.\\
*Montrer que si $p$ est un nombre premier positif, $p=a^2+b^2$, $(a,b)\in\mathbb{Z}^2$ ssi $p\equiv 1$ mod $4$ (on pourra appliquer le th\'eor\`eme de Wilson pour voir que l'\'equation $x^2\equiv -1$ mod $p$ admet une solution).\\
*\'Etudier l'anneau $\mathbb{Z}[\sqrt{2}]$.
\subsubsection{Anneaux des fonctions}
Montrer que l'ensemble des fonctions r\'eelles continues sur [0,1] est un anneau. Est-il int\`egre?\\
Montrer que l'ensemble des fonctions qui s'annulent en $0$ est un id\'eal maximal de cet anneau.\\
Trouver tous les id\'eaux maximaux de cet anneau.

\subsection{Exercices sur les anneaux}
\subsubsection{Un anneau euclidien est factoriel}\label{euclidien}
1) Montrer qu'un anneau euclidien est principal.\\
2) Montrer qu'un anneau principal est factoriel.
\subsubsection{1-ab et 1-ba}
Soit $A$ un anneau. Soient $a,b$ deux \'el\'ements dans $A$. On suppose que $1-ab$ est inversible dans $A$.\\
Montrer que $1-ba$ est inversible dans $A$. Montrer que le r\'esultat est encore valable m\^eme si $A$ n'est pas commutatif.
\subsubsection{Anneau local}
On dit qu'un anneau $R$ est local s'il ademt un et un seul id\'eal maximal $\mathfrak{m}$.\\
Montrer qu'alors $\mathfrak{m}$ est exactement l'ensemble des \'el\'ements non-inversible de $R$.\\
On admet le r\'esultat suivant, qui r\'esulte du lemme de Zorn: tout \'el\'ement non inversible est contenu dans un id\'eal maximal.

\subsection{Exemples de corps}
\subsubsection{Id\'eal maximal et corps}
1) Montrer que le quotient d'un anneau par un id\'eal maximal est un corps. R\'eciproque?\\
2) Montrer que le quotient d'un anneau par un id\'eal premier est un anneau int\`egre. R\'eciproque?
\subsubsection{Anneaux int\`gre fini}
Montrer que tout anneau int\`egre fini est un corps.
\subsubsection{Exemple d'un anneau bool\'een}
Soient $E$ un ensemble fini et $\mathcal{P}(E)$ l'ensemble de ses sous-ensembles. Munissons $E$ de:\\
Une loi additive: la diff\'erence sym\'etrique;\\
Une loi multiplicative: l'intersection.\\
Montrer que $\mathcal{P}(E)$, muni de ces lois, est un anneau.

\subsection{Exercices sur les corps}
\subsubsection{Caract\'eristique d'un corps}
Soit $K$ un corps. On regarde $n.1=1+1+\dots+1$ ($n$ fois). Deux choses peuvent se passer:\\
$\bullet$ Ou bien $\forall n, n.1\neq 0$. Montrer que dans ce cas, $K$ contient une copie de $\mathbb{R}$. On dit dans ce cas que le corps $K$ est de caract\'eristique $0$.\\
$\bullet$ Ou bien $\exists n, n.1=0$. Montrer que dans ce cas, il existe un et un seul nombre premier $p$ tel que $\forall x\in K, p.x=0$. On dit dans ce dernier cas que le corps $K$ est de caract\'eristique $p$.
\subsubsection{Sous-corps et espaces vectoriels}
Soit $k\subset K$ un sous-corps d'un corps $K$. Montrer que $K$ est un $k$-espace vectoriel.
\subsubsection{*Cardinal d'un corps fini}
Montrer que le cardinal tout corps fini s'\'ecrit comme $p^n$ avec $p$ premier positif et $n$ entier strictement positif.\\
*A-t-on la r\'eciproque?

\end{document}