\documentclass{beamer}
\usepackage[T1]{fontenc}
\usepackage[urw-garamond]{mathdesign}
\usepackage{garamondx}
\usepackage{listings}

\usetheme{Boadilla}
\usefonttheme{professionalfonts}
\usefonttheme{structuresmallcapsserif}

\setbeamertemplate{section page}
{
    \begin{centering}
    \usebeamerfont{section title}\usebeamercolor[fg]{section name}\sectionname~\MakeUppercase{\romannumeral\insertsectionnumber}\\
    \begin{beamercolorbox}[sep=12pt,center]{part title}
    \usebeamerfont{section title}\insertsection\par
    \end{beamercolorbox}
    \end{centering}
}
\defbeamertemplate{section in toc}{sections numbered roman}{%
  \leavevmode%
  \MakeUppercase{\romannumeral\inserttocsectionnumber}.\ %
  \inserttocsection\par}
\setbeamertemplate{section in toc}[sections numbered roman]

\title[DKRV14 and DMS14]{Unit volume Liouville measure on the sphere with $(\gamma,\gamma,\gamma)$-insertions: two constructions}
\subtitle{-- after David$^\circ$, Duplantier$^\bullet$, Kupiainen$^\circ$,\\Miller$^\bullet$, Rhodes$^\circ$, Sheffield$^\bullet$, Vargas$^\circ$}
\author[Yichao Huang]{Yichao Huang$^{[ENS]}$, joint with Juhan Aru$^{[ETH]}$, Xin Sun$^{[MIT]}$}
\date[IHES, 17 May 2016]{Les probabilit\'es de demain, May 2016}

\begin{document}
\AtBeginSection{\frame{\sectionpage}}

\begin{frame}
\titlepage
\end{frame}

\begin{frame}
\frametitle{Outline}
\tableofcontents
\end{frame}

\section{Insertions: classical and quantum}

\begin{frame}
\frametitle{Classical insertions}
\framesubtitle{a.k.a. conical singularities}

    \begin{center}
        $\sqrt{5} \neq 1$
    \end{center}

    \texttt{Use} the $^{power}$ of \LaTeX!
\end{frame}

\begin{frame}
\frametitle{Quantum insertions}
\framesubtitle{a.k.a. vertex operators}

    \begin{center}
        $\sqrt{5} \neq 1$
    \end{center}

    \texttt{Use} the $^{power}$ of \LaTeX!
\end{frame}

\begin{frame}
\frametitle{$\gamma$-insertions}
\framesubtitle{a.k.a. ``choose a point w.r.t. the quantum measure''}

    \begin{center}
        $\sqrt{5} \neq 1$
    \end{center}

    \texttt{Use} the $^{power}$ of \LaTeX!
\end{frame}

\section{Gaussian Free Field: decomposition}

\section{Zero mode}

\section{Reweighting factor}

\section{Theorem and consequences}

\end{document}
